\vspace{-2mm}

\cvsection{Education}

\cvevtzeronohrulenobold{2019}{\textbf{Ph.D}, Computer Science}{\textcolor{sectcol}{\hspace{16mm} Stanford University} \hspace{32.5mm} GPA: 4.0/4.0}%{Developed and evaluated an algorithm for semi automated scoring of spreadsheet data}

%\cvevtone{2019}{Ph.D, Computer Science}{\textcolor{sectcol}{Stanford University}}{Thesis: HabitLab: In-the-wild Behavior Experimentation at Scale. Advisor: Michael Bernstein. \hfill GPA: 4.0/4.0}%{Developed and evaluated an algorithm for semi automated scoring of spreadsheet data}

%\textcolor{softcol}{\hrule}

\cvevtzeronohrulenobold{2013}{\textbf{B.S.} and \textbf{M.Eng}, Computer Science}{\textcolor{sectcol}{Massachusetts Institute of Technology} \hspace{1mm} GPA: 5.0/5.0}\\ %{Developed and evaluated an algorithm for semi automated scoring of spreadsheet data}

\vspace{2mm}

%\cvevtone{2013}{B.S. and M.Eng, Computer Science}{\textcolor{sectcol}{Massachusetts Institute of Technology}}{Thesis: Multimedia for Language Learning. Advisor: Robert Miller. \hfill GPA: 5.0/5.0}

%
%\cvevtone{2013}{B.S. and M.Eng, Computer Science}{\textcolor{sectcol}{Massachusetts Institute of Technology}}{Thesis: Multimedia for Language Learning. Advisor: Robert Miller. \hfill GPA: 5.0/5.0}

%\textcolor{softcol}{\hrule}

%
%\cvevent{2012 - 2015}{Master Studies Digital Media}{University of Bremen}{Inter-cultural classes in English, covering special topics in computer science and design}{Professionalized in research methods, software development and e-assessment}

%\textcolor{softcol}{\hrule}

%
%\cvevent{2009 - 2010}{Semester Abroad}{University of Melbourne}{Mastered six months of study and trans-cultural experience in Melbourne, Australia}{Finished machine programming, information visualization, professional essay writing}


%============================================================================%
%
% CV SECTIONS AND EVENTS (MAIN CONTENT)
%
%============================================================================%

%---------------------------------------------------------------------------------------
% EXPERIENCE
%----------------------------------------------------------------------------------------
\cvsection{Work Experience}

\cvevtzeronohrule{Mar 2022 - present}{Senior Research Scientist}{\textcolor{sectcol}{Google} \hspace{1mm} Mountain View}\\
Working on LLM post-training (Gemini), generating synthetic datasets and evaluations for improving machine translation. Published work on \href{https://aclanthology.org/2024.wmt-1.109.pdf}{inference-time techniques for improving LLM translation quality}, \href{https://arxiv.org/pdf/2411.15387}{automatic evaluation metrics}, \href{https://arxiv.org/pdf/2502.12404}{evaluation datasets}, \href{https://arxiv.org/pdf/2502.12301}{low-resource language translation}, and \href{https://arxiv.org/pdf/2305.15525}{using LLMs for health applications}.\\
% Working on applications of LLMs - specifically, generating synthetic datasets and fine-tuning LLMs to improve their performance on specialized tasks, evaluating LLM performance, and building LLM powered applications.\\

\vspace{-1mm}

%\cvevtzeronohrule{Feb 2021 - Mar 2022}{Principal Research Scientist}{\textcolor{sectcol}{Lilt} \hspace{1mm} San Francisco}\\
%\cvevtzero{Aug 2019 - Feb 2021}{Senior Research Scientist}{\hspace{3.2mm} \textcolor{sectcol}{Lilt} \hspace{1mm} San Francisco}\\
\cvevtzero{Aug 2019 - Mar 2022}{Principal Research Scientist}{\hspace{3.2mm} \textcolor{sectcol}{Lilt} \hspace{1mm} San Francisco}\\
\ifdefined\hci
As the head of HCI research, I head a team of researchers improving Lilt's interactive machine translation system.\\
Developed metrics and logging to determine how translators spend their time and predict translator performance.\\
Ran A/B tests to evaluate website translation ROI, and developed system that recommends pages to translate. \\
Improved interactive MT system speed by shifting computation client-side via TensorflowJS and heuristics.\\
Built named entity transliteration system based on Transformer architecture using tensor2tensor and Tensorflow.\\  
\else
I headed a team of researchers to evaluate and improve Lilt's interactive neural machine translation system.\\
Developed evaluation metrics for evaluating interactive machine translation systems (\href{https://aclanthology.org/2022.wmt-1.75.pdf}{WMT 2022}, \href{https://aclanthology.org/2020.amta-impact.7/}{AMTA 2020}).\\
Developed corpus and system for automatic translation error detection and correction (\href{https://aclanthology.org/2022.naacl-main.36.pdf}{NAACL 2022 best paper}).\\
%Built named entity transliteration system based on Transformer architecture.\\
Evaluated effects of human vs machine translations on website engagement (\href{https://aclanthology.org/2022.amta-research.23.pdf}{AMTA 2022 best presentation}).\\
Improved interactive machine translation speed via a novel architecture that shifts computation client-side (\href{https://patents.google.com/patent/US20230070302A1}{Patent}).\\
% Developed metrics and logging to determine how translators spend their time and predict translator performance.\\
% Ran A/B tests to evaluate website translation ROI, and developed system that recommends pages to translate. \\
\fi

\vspace{-1mm}

\ifdefined\hci
\cvevtzero{Sep 2013 - July 2019}{Graduate Researcher (Ph.D)}{\textcolor{sectcol}{Stanford University} \hspace{1mm} Advisor: Michael Bernstein}\\
Created \hypersetup{urlcolor=black}\href{https://habitlab.github.io}{HabitLab}\hypersetup{urlcolor=linkcol}, an online platform with 12,000+ active users for conducting research on personalized behavior change interventions. Published papers on behavior change (\href{https://hci.stanford.edu/publications/2021/notnow/notnowasklater.pdf}{CHI 2021}, \href{https://hci.stanford.edu/publications/2019/conservation/conservation-chi2019.pdf}{CHI 2019}, \href{https://hci.stanford.edu/publications/2018/habitlab/habitlab-cscw18.pdf}{CSCW 2018}), crowdsourcing (\href{https://hci.stanford.edu/publications/2017/crowdresearch/crowd-research-uist2017.pdf}{UIST 2017}), mining student interaction data on MOOCs (\href{https://hci.stanford.edu/publications/2016/invideo/invideo-las2016.pdf}{L@S 2016}), tools for foreign language learning (\href{http://up.csail.mit.edu/other-pubs/chi2014-smartsubs.pdf}{CHI 2014}).\\
%My thesis project was HabitLab \url{http://habitlab.stanford.edu/} an app for Chrome + Android with \textcolor{sectcol}{12,000+ daily active users} which helps users reduce time online. Published first-author papers about HabitLab at CHI 2021, CHI 2019, and CSCW 2018. I also published first-author papers at Learning at Scale 2016 (Effects of In-Video Quizzes on MOOC Lecture Viewing) and CHI 2014 (Smart Subtitles for Foreign Language Learning).\\ % See page 2 for other publications.
\else
\cvevtzero{Sep 2013 - July 2019}{Graduate Researcher (Ph.D)}{\textcolor{sectcol}{Stanford University} \hspace{1mm} Advisor: Michael Bernstein}\\
Created \hypersetup{urlcolor=black}\href{https://habitlab.github.io}{HabitLab}\hypersetup{urlcolor=linkcol}, an online platform with 12,000+ active users for conducting data science research on personalized behavior change interventions. Published papers on adaptive interventions (\href{https://hci.stanford.edu/publications/2021/notnow/notnowasklater.pdf}{CHI 2021}, \href{https://hci.stanford.edu/publications/2019/conservation/conservation-chi2019.pdf}{CHI 2019}, \href{https://hci.stanford.edu/publications/2018/habitlab/habitlab-cscw18.pdf}{CSCW 2018}), crowdsourcing (\href{https://hci.stanford.edu/publications/2017/crowdresearch/crowd-research-uist2017.pdf}{UIST 2017}), large-scale interaction data (\href{https://hci.stanford.edu/publications/2016/invideo/invideo-las2016.pdf}{L@S 2016}), NLP for language learning (\href{http://up.csail.mit.edu/other-pubs/chi2014-smartsubs.pdf}{CHI 2014}).\\
%My thesis project was HabitLab \url{http://habitlab.stanford.edu/} an app for Chrome + Android with \textcolor{sectcol}{12,000+ daily active users} which helps users reduce time online. Published first-author papers about HabitLab at CHI 2021, CHI 2019, and CSCW 2018. I also published first-author papers at Learning at Scale 2016 (Effects of In-Video Quizzes on MOOC Lecture Viewing) and CHI 2014 (Smart Subtitles for Foreign Language Learning).\\ % See page 2 for other publications.
\fi


\vspace{-1mm}


%
\cvevtone{Summer 2015}{Research Intern}{\textcolor{sectcol}{Microsoft Research} \hspace{1mm} Redmond}{Designed and built an educational social feed app usable by pre-literate children. Published at \href{https://www.microsoft.com/en-us/research/wp-content/uploads/2016/10/edufeed.pdf}{CSCW 2017}.} % Manager: Merrie Morris

\vspace{-1mm}


\cvevtone{Summer 2014}{Research Intern}{\textcolor{sectcol}{Microsoft Research} \hspace{1mm} Beijing}{Built QuizCram, a quiz-driven MOOC lecture viewer that improves learning outcomes. Presented at \href{https://hci.stanford.edu/publications/2015/quizcram/quizcram-chi2015.pdf}{CHI 2015.}} % Manager: Darren Edge

\vspace{-1mm}


\cvevtone{Summer 2013}{Software Engineering Intern}{\textcolor{sectcol}{Google} \hspace{1mm} Mountain View}{Developed a machine learning system for detecting taps on the phone bezel, for use in Android input methods.} % Manager: Shumin Zhai

\vspace{-1mm}


\cvevtone{Summer 2012}{Software Engineering Intern}{\textcolor{sectcol}{Google} \hspace{1mm} Mountain View}{Developed an NLP system to automatically generate glossaries from book text. Patent granted \href{https://patents.google.com/patent/US9483460B2}{US9483460B2}.} % Manager: Ulas Kirazci

\vspace{-1mm}


\cvevtone{Summer 2011}{Software Engineering Intern}{\textcolor{sectcol}{Google} \hspace{1mm} Mountain View}{Developed a machine learning system to predict the quality of user reviews, now deployed on Google Play.} % Manager: Ulas Kirazci

\vspace{-1mm}


\cvevtone{Summer 2010}{Software Development Engineer Intern}{\textcolor{sectcol}{Microsoft} \hspace{1mm} Redmond}{Built the IntelliSense code completion system for a scientific computing language, and contributed to its compiler.} % Manager: Davis Walker

%--------------------------------------------------------------------------------------------------
% ARTIFICIAL FOOTER (fancy footer cannot exceed linewidth) 
%--------------------------------------------------------------------------------------------------


%\vspace{2mm}

%\cvsection{Publications in Academic Conferences and Journals}
%\cvsection{Select Journal and Conference Papers}
\cvsection{Select Publications}

%\textbf{Geza Kovacs}, Daniel Deutsch, Markus Freitag.
``\href{https://aclanthology.org/2024.wmt-1.109.pdf}{Mitigating Metric Bias in Minimum Bayes Risk Decoding.}'' \emph{Proceedings of the Ninth Conference on Machine Translation (WMT). 2024.}\\

% {\small Jessy Lin,} \textbf{Geza Kovacs}, {\small Aditya Shastry, Jorn Wuebker, John DeNero.}
``\href{https://aclanthology.org/2022.naacl-main.36.pdf}{Automatic Correction of Human Translations}'' \emph{Proceedings of the 2022 Conference of the North American Chapter of the Association for Computational Linguistics. 2022}. \textbf{Best Paper Award}, Best New Task, and Best New Resource.\\ % \textbf{Best Paper Award} in 2 categories: Best New Task and Best New Resource.\\

% {\small Mara Finkelstein, Daniel Deutsch, Parker Riley, Juraj Juraska} \textbf{Geza Kovacs}, {\small Markus Freitag.}
``\href{https://arxiv.org/pdf/2411.15387}{From Jack of All Trades to Master of One: Specializing LLM-based Autoraters to a Test Set.}'' Under review, 2025.\\

% {\small Daniel Deutsch, Eleftheria Briakou, Isaac Caswell, Mara Finkelstein, Rebecca Galor, Juraj Juraska,} \textbf{Geza Kovacs}, {\small Alison Lui, Ricardo Rei, Jason Riesa, Shruti Rijhwani, Parker Riley, Elizabeth Salesky, Firas Trabelsi, Stephanie Winkler, Biao Zhang, Markus Freitag.}
``\href{https://arxiv.org/pdf/2502.12404}{WMT24++: Expanding the Language Coverage of WMT24 to 55 Languages \& Dialects.}'' Under review, 2025.\\

% {\small Xin Liu, Daniel McDuff,} \textbf{Geza Kovacs}, {\small Isaac Galatzer-Levy, Jacob Sunshine, Jiening Zhan, Ming-Zher Poh, Shun Liao, Paolo Di Achille, Shwetak Patel.}
``\href{https://arxiv.org/pdf/2305.15525}{Large Language Models are Few-Shot Health Learners.}'' Under review, 2025.\\

% {\small Isaac Caswell, Elizabeth Nielsen, Jiaming Luo, Colin Cherry,} \textbf{Geza Kovacs}, {\small Hadar Shemtov, Partha Talukdar, Dinesh Tewari, Baba Mamadi Diane, Koulako Moussa Doumbouya, Djibrila Diane, Solo Farabado Cissé.}
``\href{https://arxiv.org/pdf/2502.12301}{SMOL: Professionally translated parallel data for 115 under-represented languages.}'' Under review, 2025.\\

% {\small Mike A. Merrill, Akshay Paruchuri, Naghmeh Rezaei,} \textbf{Geza Kovacs}, {\small Javier Perez, Yun Liu, Erik Schenck, Nova Hammerquist, Jake Sunshine, Shyam Tailor, Kumar Ayush, Hao-Wei Su, Qian He, Cory Y. McLean, Mark Malhotra, Shwetak Patel Jiening Zhan, Tim Althoff, Daniel McDuff, Xin Liu}
``\href{https://arxiv.org/pdf/2406.06464}{Transforming Wearable Data into Health Insights using Large Language Model Agents.}'' Under review, 2025.\\

% \textbf{Geza Kovacs} {\small and John DeNero.}
``\href{https://aclanthology.org/2022.amta-research.23.pdf}{Measuring the Effects of Human and Machine Translation on Website Engagement.}'' \emph{Proceedings of the 15th Conference of the Association for Machine Translation in the Americas (Research Track). 2022}. \textbf{Best Presentation Award.}\\

% {\small Francisco Casacuberta, George Foster, Guoping Huang, Philipp Koehn,} \textbf{Geza Kovacs}, {\small Lemao Liu, Shuming Shi, Taro Watanabe, Chengqing Zong.}
``\href{https://aclanthology.org/2022.wmt-1.75.pdf}{Findings of the Word-Level AutoCompletion Shared Task in WMT 2022.}'' \emph{Proceedings of the Seventh Conference on Machine Translation (WMT). 2022.}\\

% {\small Samuel Läubli, Patrick Simianer, Joern Wuebker,} \textbf{Geza Kovacs}, {\small Rico Sennrich, Spence Green.}
``\href{https://arxiv.org/pdf/2011.05978.pdf}{The Impact of Text Presentation on Translator Performance.}'' \emph{Target: International Journal of Translation Studies, 2021}.\\

% \textbf{Geza Kovacs}, {\small Zhengxuan Wu, Michael Bernstein.}
``\href{https://hci.stanford.edu/publications/2021/notnow/notnowasklater.pdf}{Not Now, Ask Later: Users Weaken Their Behavior Change Regimen Over Time, But Expect To Re-Strengthen It Imminently.}'' \emph{ACM annual conference on Human Factors in Computing Systems (CHI) 2021}. Acceptance rate: 23\%.\\ %

% \textbf{Geza Kovacs}, {\small Drew Mylander Gregory, Zilin Ma, Zhengxuan Wu, Golrokh Emami, Jacob Ray, Michael Bernstein.}
``\href{https://hci.stanford.edu/publications/2019/conservation/conservation-chi2019.pdf}{Conservation of Procrastination: Do Productivity Interventions Save Time Or Just Redistribute It?}'' \emph{ACM annual conference on Human Factors in Computing Systems (CHI) 2019}. Acceptance rate: 23.8\%.\\ %

%\pagebreak

% \textbf{Geza Kovacs}, {\small Zhengxuan Wu, Michael Bernstein.}
``\href{https://hci.stanford.edu/publications/2018/habitlab/habitlab-cscw18.pdf}{Rotating Online Behavior Change Interventions Increases Effectiveness But Also Increases Attrition.}'' \emph{ACM annual conference on Computer-Supported Cooperative Work and Social Computing (CSCW) 2018}. Acceptance rate: 26\%.\\

% {\small Rajan Vaish, Neil Gaikwad,} \textbf{Geza Kovacs}, {\small Andreas Veit, Ranjay Krishna, Imanol Arrieta Ibarra, Camelia Simoiu, Michael Wilber, Serge Belongie, Sharad Goel, James Davis, Michael Bernstein.}
``\href{https://hci.stanford.edu/publications/2017/crowdresearch/crowd-research-uist2017.pdf}{Crowd Research: Open and Scalable University Laboratories.}'' \emph{ACM Symposium on User Interface Software and Technology (UIST) 2017}. Acceptance rate: 22\%.\\

% {\small Kiley Sobel,} \textbf{Geza Kovacs}, {\small Galen McQuillen, Andrew Cross, Nirupama Chandrasekaran, Nathalie Riche, Ed Cutrell, Meredith Morris.}
``\href{https://www.microsoft.com/en-us/research/wp-content/uploads/2016/10/edufeed.pdf}{EduFeed: A Social Feed to Engage Preliterate Children in Educational Activities.}'' \emph{ACM annual conference on Computer-Supported Cooperative Work and Social Computing (CSCW) 2017}. Acceptance rate: 35\%.\\

% \textbf{Geza Kovacs}.
``\href{https://hci.stanford.edu/publications/2016/invideo/invideo-las2016.pdf}{Effects of In-Video Quizzes on MOOC Lecture Viewing.}'' \emph{ACM annual conference on Learning at Scale (L@S) 2016}. Acceptance rate: 22\%.\\

% \textbf{Geza Kovacs} {\small and Robert C. Miller.}
``\href{http://up.csail.mit.edu/other-pubs/chi2014-smartsubs.pdf}{Smart Subtitles for Vocabulary Learning.}'' \emph{ACM annual conference on Human Factors in Computing Systems (CHI) 2014}. Acceptance rate: 23\%.\\

% \cvsection{Workshop and Demo Papers}

% {\small Stanford Crowd Research,} \textbf{Geza Kovacs}, {\small Rajan Vaish, Michael Bernstein.} ``\href{https://hci.stanford.edu/publications/2015/crowdresearch/daemo-uist.pdf}{Daemo: A Self-Governed Crowdsourcing Marketplace.}'' \emph{ACM Symposium on User Interface Software and Technology (UIST) 2015, Poster}.\\

% \textbf{Geza Kovacs}. ``\href{https://hci.stanford.edu/publications/2015/feedlearn/feedlearn-chi2015.pdf}{FeedLearn: Using Facebook Feeds for Microlearning.}'' \emph{ACM annual conference on Human Factors in Computing Systems (CHI) 2015, Extended Abstracts}.\\ % Acceptance rate: 45\%

% \textbf{Geza Kovacs}. ``\href{https://hci.stanford.edu/publications/2015/quizcram/quizcram-chi2015.pdf}{QuizCram: A Question-Driven Video Studying Interface.}'' \emph{ACM annual conference on Human Factors in Computing Systems (CHI) 2015, Extended Abstracts}.\\

% {\small Joseph Jay Williams,} \textbf{Geza Kovacs}, {\small Caren Walker, Samuel G Maldonado, Tania Lombrozo.} ``\href{https://hci.stanford.edu/publications/2014/explain/explain-chi2014.pdf}{Learning Online via Prompts to Explain.}'' \emph{ACM annual conference on Human Factors in Computing Systems (CHI) 2014, Extended Abstracts}.\\

% \textbf{Geza Kovacs} {\small and Robert C. Miller.} ``\href{http://up.csail.mit.edu/other-pubs/uist2013-mangareader.pdf}{Foreign Manga Reader: Learn Grammar and Pronunciation while Reading Comics.}'' \emph{ACM Symposium on User Interface Software and Technology (UIST) 2013, Demo}.\\

% \textbf{Geza Kovacs}. ``\href{http://up.csail.mit.edu/other-pubs/chi2013-smartsubs.pdf}{Smart Subtitles for Language Learning.}'' \emph{ACM annual conference on Human Factors in Computing Systems (CHI) 2013, Extended Abstracts}.\\ %\\
% %\url{http://groups.csail.mit.edu/uid/other-pubs/chi2013-smartsubs.pdf}

% \textbf{Geza Kovacs}. ``\href{http://up.csail.mit.edu/other-pubs/chi2012-screenshots-for-translation-context.pdf}{ScreenMatch: providing context to software translators by displaying screenshots.}'' \emph{ACM annual conference on Human Factors in Computing Systems (CHI) 2012, Extended Abstracts}.\\ %\\
% %\url{http://groups.csail.mit.edu/uid/other-pubs/chi2012-screenshots-for-translation-context.pdf}

% %\cvsection{Preprints}


%\vspace{-5mm}

\thispagestyle{secondstyle}

\vspace{-2mm}

\cvsection{Patents}

{\small Tania Bedrax-Weiss,} \textbf{Geza Kovacs}, {\small Ulas Kirazci.} ``\href{https://patents.google.com/patent/US9483460B2}{Automated formation of specialized dictionaries.}'' \hypersetup{urlcolor=black}\href{https://patents.google.com/patent/US9483460B2}{US9483460B2}\hypersetup{urlcolor=linkcol}. Filed 10/2013, Published 11/2016, Expires 01/2034.\\

{\small Meredith Morris, Nathalie Henry Riche, Edward B . Cutrell, Andrew C . Cross, Natasa Milic, Nirupama Chandrasekaran, Galen McQuillen, Kiley Sobel,} \textbf{Geza Kovacs}. ``\href{https://patents.google.com/patent/US20180068578A1}{Presenting educational activities via an extended social media feed.}'' . Filed 09/2016, Published 03/2018.\\ % \hypersetup{urlcolor=black}\href{https://patents.google.com/patent/US20180068578A1}{US20180068578A1}\hypersetup{urlcolor=linkcol}

\vspace{-2mm}

\pagebreak

\cvsection{Invited Keynote Talks}

\textbf{Geza Kovacs}. ``\href{https://aclanthology.org/2020.amta-impact.7/}{Predictive Translation Memory in the Wild: A Study of Interactive Machine Translation Use on Lilt.}'' \emph{Association for Machine Translation in the Americas (AMTA) Workshop on the Impact of Machine Translation 2020}.\\

\vspace{-2mm}

\cvsection{Open Source Projects}

% \url{http://unetbootin.github.io/}
\textbf{UNetbootin} (LiveUSB Creator) \hfill \url{https://en.wikipedia.org/wiki/UNetbootin}\\ %\hfill \textsl{January 2007 -- now}\\
\textcolor{sectcol}{40 million downloads.} UNetbootin creates bootable USB flash drives for various (50+) Linux distributions.\\ %This work has been accepted into the official package repositories for Debian, Ubuntu, Fedora, openSUSE, Gentoo, and other major distributions. \\
% \hspace{3mm}  

\textbf{Wubi} (Ubuntu Installer for Windows) \hfill \url{https://en.wikipedia.org/wiki/Wubi_(software)}\\ %\hfill \textsl{November 2006 -- August 2007}\\
\textcolor{sectcol}{Now part of Ubuntu.} Built the first versions of Wubi, which allows Ubuntu to be installed from Windows.\\ % This work is now part of Ubuntu. \\

\textbf{HabitLab} (In-the-wild Behavior Change Research Platform) \hfill \url{https://habitlab.stanford.edu}\\ %\hfill \textsl{November 2006 -- August 2007}\\
\textcolor{sectcol}{12,000+ daily active users.} I built HabitLab over my Ph.D, and it is still used for research at Stanford Medical School.\\ % This work is now part of Ubuntu. \\

% Reviewer for \textbf{CHI} (2015, 2018, 2019, 2021), \textbf{UIST} (2017, 2018), \textbf{CSCW} (2021), \textbf{IMWUT} (2019).\\


\vspace{-2mm}

%\pagebreak
\cvsection{Select Awards and Honors}

Best Presentation Award, AMTA 2022 \hfill \textcolor{black}{2018}\\ %-2016\\
Best Paper Award, NAACL 2022 \hfill \textcolor{black}{2018}\\ %-2016\\
Stanford Human-Centered AI Grant (for my research project HabitLab) \hfill \textcolor{black}{2018}\\ %-2016\\
National Defense Science and Engineering Graduate Fellowship \hfill \textcolor{black}{2013}\\ %-2016\\
National Science Foundation Graduate Research Fellowship \hfill \textcolor{black}{2013}\\ %(declined in favor of NDSEG), 2013\\
% Finalist and Honorable Mention, MIT Web Programming Competition (6.470), 2013\\ % (Project: PsetParty)
%1$^{\textrm{st}}$ place, Most Useful, ACM UIST (User Interface Software and Technology) Student Innovation Contest \hfill \textcolor{sectcol}{2012}\\
1$^{\textrm{st}}$ place, ACM UIST (User Interface Software and Technology) Student Innovation Contest \hfill \textcolor{black}{2012}\\
1$^{\textrm{st}}$ place, ACM CHI (Human Factors in Computing Systems) Student Research Competition \hfill \textcolor{black}{2012}\\
%1$^{\textrm{st}}$ place award for my project, InstantKaraoke, at Boston Music Hack Day \hfill \textcolor{sectcol}{2012}\\ % (Project: InstantKaraoke)
Phi Beta Kappa (top 10\% of students at MIT), Tau Beta Pi (top 12.5\% of Engineering students at MIT) \hfill \textcolor{black}{2012}\\
% Phi Beta Kappa (top 10\% of students at MIT) \hfill \textcolor{black}{2012}\\
% Tau Beta Pi (top 12.5\% of Engineering students at MIT), Eta Kappa Nu (top 25\% of EECS students at MIT) \hfill \textcolor{black}{2011}\\
% 1$^{\textrm{st}}$ place, MIT Autonomous Robotics Competition (MASLAB) \hfill \textcolor{black}{2010}\\


\cvsection{Academic Conference Reviewing and Committees}

Organizing Committee, WMT 2022 Shared Task on Word-Level Auto-Completion \hfill 2022

Program Committee, EACL 2021 Bridging HCI and NLP Workshop \hfill 2021

Reviewer, ACM Conference on Human Factors in Computing Systems (CHI) \hfill \textcolor{black}{2015, 2018-2019, 2021-2024}

Reviewer, ACM Conference on Designing Interactive Systems (DIS) \hfill \textcolor{black}{2023}

Reviewer, ACM Conference on Computer-Supported Cooperative Work and Social Computing (CSCW) \hfill \textcolor{black}{2021}

Reviewer, ACM Transactions on Computer-Human Interaction (TOCHI) \hfill \textcolor{black}{2022}

Reviewer, ACM Interactive, Mobile, Wearable and Ubiquitous Technologies (IMWUT) \hfill \textcolor{black}{2019}

Reviewer, ACM Symposium on User Interface Software and Technology (UIST) \hfill \textcolor{black}{2017-2018}\\

%\pagebreak

\cvsection{Researchers Managed}

Sai Gouravajhala, Senior Research Scientist at Lilt. \hfill August 2020 -- March 2022

Hannah Yan, Senior Data Scientist at Lilt. \hfill September 2020 -- March 2022

Jordan Huffaker, Research Intern at Lilt. Now a Ph.D student at University of Michigan. \hfill Summer 2021

Jessy Lin, Research Engineer at Lilt. Now a Ph.D student at UC Berkeley. \hfill August 2019 -- August 2020

Ming-Chang Chiu, Data Science Intern at Lilt. Now a Ph.D student at USC. \hfill Summer 2020\\

% Zhengxuan Wu, MS student researcher at Stanford. \hfill Fall 2018 -- Spring 2019

% Zilin Ma, undergraduate researcher at Stanford. Now a Ph.D student at Harvard. \hfill Summer 2018\\

% Drew Mylander Gregory, undergraduate researcher at Stanford. \hfill Summer 2018

% Golrokh Emami, undergraduate researcher at Stanford. \hfill Summer 2018\\

% Jacob Ray, undergraduate researcher at Stanford. \hfill Summer 2018\\


\cvsection{Teaching Experience}

Understanding Users (CS 377U) -- Teaching Assistant, at Stanford \hfill \textcolor{black}{Spring 2019}

Human Computer Interaction Research (CS 376) -- Teaching Assistant, at Stanford \hfill \textcolor{black}{Fall 2018}
%Helped write and grade assignments, lead discussions, and manage the course infrastructure.

Natural Language Processing (6.863) -- Teaching Assistant, at MIT \hfill \textcolor{black}{Fall 2012}\\
%Helped write and grade assignments, and managed the course infrastructure. I developed new tools to make the assignment grading process faster, semi-automatic, and paper-free.

% \textbf{Introduction to C++ IAP (6.096)} -- Instructor, at MIT \hfill \textcolor{black}{January 2011}

%Gave lectures, helped write and grade assignments, and helped students in lab for a student-run, for-credit introductory C++ course. The teaching materials I produced have been made available on OpenCourseWare:
% My lectures and teaching materials for this course are available on MIT OpenCourseWare:

% \begin{footnotesize}
% \url{http://ocw.mit.edu/courses/electrical-engineering-and-computer-science/6-096-introduction-to-c-january-iap-2011}
% \end{footnotesize}\\

% \textbf{MASLAB Mobile Autonomous Systems Lab (6.186)} -- Software Director, at MIT \hfill \textcolor{black}{January 2011}

% Gave lectures on computer vision and managed the software stack for MIT's autonomous robotics competition.

% As the software director for this student-led autonomous robotics competition, I gave lectures on computer vision and control algorithms, managed the software stack for the competition, and helped students in lab.


\cvsection{Skills and Technologies}

\textbf{Programming Languages}: Python, JavaScript, C, C++, Java, TypeScript, R, C\#, Ruby, Scala, Haskell, Bash, SQL %, Scala, C\#, Ruby, CoffeeScript, Haskell, Bash

\textbf{Machine Learning + Deep Learning}: Tensorflow, PyTorch, TensorflowJS, Keras, scikit-learn, xgboost, MLFlow %Reinforcement Learning %(RNN/LSTM/CNN/GAN) % (RNN/CNN/GAN) %, NLP %, PCFG % Weka % sklearn % scikit-learn
%\textbf{Machine Learning} and \textbf{Data Science}: PyTorch, TensorFlow, Python, Numpy, SciPy, Pandas, Jupyter, R, rpy2

\textbf{Natural Language Processing + Machine Translation}: SpaCy, tensor2tensor, fairseq, HuggingFace, LASER, NLTK %LASER, NLTK % NLTK LASER % BERT, LASER sentence embeddings, word embeddings %Transformer, NLTK, skip-grams, word2vec, GloVe, Attention Networks, HMM, PCFG % Probabalistic Context-Free Grammars, Parsing % (RNN/CNN/GAN) %, NLP %, PCFG % Weka % sklearn % scikit-learn
%\textbf{Machine Learning} and \textbf{Data Science}: PyTorch, TensorFlow, Python, Numpy, SciPy, Pandas, Jupyter, R, rpy2

%\textbf{Data Mining}: Jupyter, NumPy, SciPy, Pandas, NLTK, NetworkX, MapReduce, Mongo, SQL, ggplot2, Plotly

\textbf{Data Science + Visualization}: NumPy, SciPy, Pandas, Jupyter, RStudio, Plotly, D3.js, Superset, Hadoop, MapReduce % Spark NetworkX %H2O, SQL, NoSQL (MongoDB/Redis) % CUDA

% \textbf{Data Analysis and Visualization}: Jupyter, RStudio, Plotly, D3.js, ggplot2, matplotlib, seaborn, bokeh, streamlit %, Clustering, Sentiment Analysis % Chartjs, matplotlib

% \textbf{Quantitative UX Research}: Mixed models, Survival analysis, A/B testing, Experiment design, ANOVA, statsmodels %, Multi-armed bandits, mTurk %, Crowdsourcing %Crowdsourcing
% Pandas, ANOVA, Cox regression, Cox regression, Pandas, Survival analysis, Cox regression, Multi-armed bandit, Experiment Design, Multi-armed bandits


%, CUDA

% CUDA

%\textbf{Backend Development}: Node.js (Express, Koa), Flask, MongoDB, PostgreSQL, Redis, Vagrant, EC2, GCloud

% \textbf{Backend Development}: Node.js, Flask, Docker, Kubernetes, MySQL, MongoDB, Redis, AWS EC2, Google Cloud %, Vagrant %, GCloud %, EC2

%\vspace{-4mm}

%\textbf{Mobile Development}: Cross-platform JS (Cordova, NativeScript), Android (Java), Responsive Web Design

%\textbf{Web and Mobile:} HTML/CSS/JS, Polymer, D3.js, Android (Java, Cordova, NativeScript), Responsive Design

%\textbf{Backend Development}: Node.js (Express, Koa), Flask, MongoDB, PostgreSQL, Redis, Vagrant, EC2, GCloud

%\vspace{-4mm}

%\textbf{Web and Mobile}: HTML, CSS, JS, Polymer, D3.js, Plotly, CoffeeScript, Webpack, SystemJS, React, Flow

%\textbf{Web Development}: HTML, CSS, JS, Polymer, D3.js, Plotly, CoffeeScript, Webpack, SystemJS, React, Flow
%\textbf{Web Development}: HTML, CSS, JavaScript, TypeScript, Node.js, Express, Flask, Polymer, React, MongoDB, CoffeeScript, Sass

\textbf{Web + Mobile Development}: HTML, CSS, React, AngularJS, NodeJS, Express, Flask, MySQL, Docker, Android %, Polymer React Native, Cordova, NativeScript %, Responsive Design
%Mobile Development: Cross-platform JS (Cordova, NativeScript), Android (Java), Responsive Web Design
% Java

%\vspace{-4mm}

\textbf{Languages}: Fluent: English, Chinese (Mandarin), Hungarian. Intermediate: Japanese, Vietnamese, Spanish. %\\

%\vspace{-1mm}

% \cvsection{Select Coursework}

Deep Learning (Stanford CS230), Natural Language Processing (MIT 6.864+6.863), Data Science (Stanford CS224w), Machine Learning (MIT 6.034), Statistical Models (MIT 6.804), Statistics (MIT 18.440), Linear Algebra (MIT 18.700), UX Design (MIT 6.803+MAS.672), Linguistics (MIT 24.900), Bioinformatics (MIT 6.047), Algorithms (MIT 6.006+6.046)\\ % Security (MIT 6.857), Bioinformatics (MIT 6.047)  % Bioinformatics (MIT 6.047) Compilers (Stanford CS143), Computational Music (Stanford CS275a) Algorithms (MIT 6.006+6.046)

\pagebreak

%\cvsection{Selected Press}

\textcolor{sectcol}{\textbf{HabitLab}}\\

\textbf{WIRED} - The HabitLab Browser Extension Curbs Your Time Wasted on the Web. \hfill January 2019\\
\url{https://www.wired.com/story/habitlab-browser-extension/}\\

\textbf{Lifehacker} - Prevent Procrastination With This Chrome Extension. \hfill February 2019\\
\url{https://lifehacker.com/prevent-procrastination-with-this-chrome-extension-1832723418}\\

\textbf{The New York Times} - Finding It Hard to Focus? Maybe It’s Not Your Fault. \hfill August 2018\\
\url{https://www.nytimes.com/2018/08/14/style/how-can-i-focus-better.html}\\

%\pagebreak

\textbf{Lifehacker} - Be More Mindful of the Time You Waste Online With HabitLab. \hfill August 2018\\
\url{https://lifehacker.com/be-more-mindful-of-the-time-you-waste-online-with-habit-1828118354}\\

\textbf{WIRED} - The Chrome Extensions We Can't Live Without. \hfill February 2018\\
\url{https://www.wired.com/story/best-chrome-extensions/}\\

\textbf{How-To Geek} - HabitLab Subtly Helps You Change Bad Online Habits. \hfill August 2018\\
\url{https://www.howtogeek.com/fyi/free-download-habitlab-subtly-helps-you-change-bad-online-habits/}\\

\textbf{The Stanford Daily} - HabitLab browser extension aims to help users \hfill March 2019\\
regain control of their online browsing behavior.\\
\begin{footnotesize}
\url{https://www.stanforddaily.com/2019/03/13/habitlab-browser-extension-aims-to-help-users-regain-control-of-their-online-browsing-behavior/}
\end{footnotesize}\\

\vspace{-4mm}

\textbf{Entrepreneur} - Use These Strategies to Maximize Productivity Without Inventing an Extra Weekday. \hfill May 2018\\
\url{https://www.entrepreneur.com/article/312764}\\

%\textbf{PCMag} The Best Free Google Chrome Extensions, July 2020.\\
%\url{https://www.pcmag.com/news/the-100-best-free-google-chrome-extensions}\\

%\vspace{-4mm}

\textbf{Tencent News} - (Chinese) \begin{CJK*}{UTF8}{gbsn}这款斯坦福大学的工具,让你远离加班,提升200%效率\end{CJK*} \hfill March 2019\\
\url{https://new.qq.com/omn/20190311/20190311A0BBRU.html}\\

%\pagebreak

\textcolor{sectcol}{\textbf{Crowd Research / Daemo}}\\

\textbf{Stanford University News} - A Stanford-led platform for crowdsourced research \hfill October 2017\\
gives experience to global participants\\
\url{https://news.stanford.edu/2017/10/23/crowdsourced-research-gives-experience-global-participants/}\\

\textbf{WIRED} - Amazon's Turker Crowd Has Had Enough. \hfill August 2017\\
\url{https://www.wired.com/story/amazons-turker-crowd-has-had-enough/}\\

\textcolor{sectcol}{\textbf{UNetbootin}}\\

\textbf{Forbes} - How To Try Linux Without Making Any Changes To Your PC. \hfill September 2018\\
\begin{small}
\url{https://www.forbes.com/sites/jasonevangelho/2018/09/18/how-to-safely-try-linux-on-your-mac-or-windows-pc/}
\end{small}\\

\textbf{PCWorld} - Create a Bootable Linux Flash Drive in Three Easy Steps. \hfill February 2012\\
\url{https://www.pcworld.com/article/249870/create_a_bootable_linux_flash_drive_in_three_easy_steps.html}\\

\textbf{Lifehacker} - The Complete Guide to Saving Your Windows System with a Thumb Drive. \hfill March 2010\\
\url{https://lifehacker.com/the-complete-guide-to-saving-your-windows-system-with-a-5504531}\\

%Installing Ubuntu on an old netbook with hair tearing and profanity. Network World, August 2014.\\
%\begin{footnotesize}
%\url{https://www.networkworld.com/article/2465150/installing-ubuntu-on-an-old-netbook-with-hair-tearing-and-profanity.html}
%\end{footnotesize}\\

%Linux on the NUC: Using Ubuntu, Mint, Fedora, and the SteamOS beta. Ars Technica, February 2014.\\
%\begin{small}
%\url{https://arstechnica.com/gadgets/2014/02/linux-on-the-nuc-using-ubuntu-mint-fedora-and-the-steamos-beta/}
%\end{small}\\

\textcolor{sectcol}{\textbf{Wubi}}\\

\textbf{Ars Technica} - Wubi arrives: a look at Ubuntu 8.04 alpha 5. \hfill February 2008\\
\url{https://arstechnica.com/information-technology/2008/02/wubi-arrives-a-look-at-ubuntu-8-04-alpha-5/}\\

\textbf{Lifehacker} - Install Ubuntu on a Windows Netbook, No Partitioning Needed. \hfill May 2008\\
\url{https://lifehacker.com/install-ubuntu-on-a-windows-netbook-no-partitioning-ne-5542387}\\

\textbf{PCWorld} - The Ubuntu guide for displaced Windows users. \hfill March 2013\\
\url{https://www.pcworld.com/article/2030132/the-ubuntu-guide-for-displaced-windows-users.html}\\
\cvsection{Selected Press}

\textcolor{sectcol}{\textbf{HabitLab}}\\

\textbf{WIRED} - The HabitLab Browser Extension Curbs Your Time Wasted on the Web. \hfill January 2019\\
\url{https://www.wired.com/story/habitlab-browser-extension/}\\

\textbf{Lifehacker} - Prevent Procrastination With This Chrome Extension. \hfill February 2019\\
\url{https://lifehacker.com/prevent-procrastination-with-this-chrome-extension-1832723418}\\

\textbf{The New York Times} - Finding It Hard to Focus? Maybe It’s Not Your Fault. \hfill August 2018\\
\url{https://www.nytimes.com/2018/08/14/style/how-can-i-focus-better.html}\\

%\pagebreak

\textbf{Lifehacker} - Be More Mindful of the Time You Waste Online With HabitLab. \hfill August 2018\\
\url{https://lifehacker.com/be-more-mindful-of-the-time-you-waste-online-with-habit-1828118354}\\

\textbf{WIRED} - The Chrome Extensions We Can't Live Without. \hfill February 2018\\
\url{https://www.wired.com/story/best-chrome-extensions/}\\

\textbf{How-To Geek} - HabitLab Subtly Helps You Change Bad Online Habits. \hfill August 2018\\
\url{https://www.howtogeek.com/fyi/free-download-habitlab-subtly-helps-you-change-bad-online-habits/}\\

\textbf{The Stanford Daily} - HabitLab browser extension aims to help users \hfill March 2019\\
regain control of their online browsing behavior.\\
\begin{footnotesize}
\url{https://www.stanforddaily.com/2019/03/13/habitlab-browser-extension-aims-to-help-users-regain-control-of-their-online-browsing-behavior/}
\end{footnotesize}\\

\vspace{-4mm}

\textbf{Entrepreneur} - Use These Strategies to Maximize Productivity Without Inventing an Extra Weekday. \hfill May 2018\\
\url{https://www.entrepreneur.com/article/312764}\\

%\textbf{PCMag} The Best Free Google Chrome Extensions, July 2020.\\
%\url{https://www.pcmag.com/news/the-100-best-free-google-chrome-extensions}\\

%\vspace{-4mm}

\textbf{Tencent News} - (Chinese) \begin{CJK*}{UTF8}{gbsn}这款斯坦福大学的工具,让你远离加班,提升200%效率\end{CJK*} \hfill March 2019\\
\url{https://new.qq.com/omn/20190311/20190311A0BBRU.html}\\

%\pagebreak

\textcolor{sectcol}{\textbf{Crowd Research / Daemo}}\\

\textbf{Stanford University News} - A Stanford-led platform for crowdsourced research \hfill October 2017\\
gives experience to global participants\\
\url{https://news.stanford.edu/2017/10/23/crowdsourced-research-gives-experience-global-participants/}\\

\textbf{WIRED} - Amazon's Turker Crowd Has Had Enough. \hfill August 2017\\
\url{https://www.wired.com/story/amazons-turker-crowd-has-had-enough/}\\

\textcolor{sectcol}{\textbf{UNetbootin}}\\

\textbf{Forbes} - How To Try Linux Without Making Any Changes To Your PC. \hfill September 2018\\
\begin{small}
\url{https://www.forbes.com/sites/jasonevangelho/2018/09/18/how-to-safely-try-linux-on-your-mac-or-windows-pc/}
\end{small}\\

\textbf{PCWorld} - Create a Bootable Linux Flash Drive in Three Easy Steps. \hfill February 2012\\
\url{https://www.pcworld.com/article/249870/create_a_bootable_linux_flash_drive_in_three_easy_steps.html}\\

\textbf{Lifehacker} - The Complete Guide to Saving Your Windows System with a Thumb Drive. \hfill March 2010\\
\url{https://lifehacker.com/the-complete-guide-to-saving-your-windows-system-with-a-5504531}\\

%Installing Ubuntu on an old netbook with hair tearing and profanity. Network World, August 2014.\\
%\begin{footnotesize}
%\url{https://www.networkworld.com/article/2465150/installing-ubuntu-on-an-old-netbook-with-hair-tearing-and-profanity.html}
%\end{footnotesize}\\

%Linux on the NUC: Using Ubuntu, Mint, Fedora, and the SteamOS beta. Ars Technica, February 2014.\\
%\begin{small}
%\url{https://arstechnica.com/gadgets/2014/02/linux-on-the-nuc-using-ubuntu-mint-fedora-and-the-steamos-beta/}
%\end{small}\\

\textcolor{sectcol}{\textbf{Wubi}}\\

\textbf{Ars Technica} - Wubi arrives: a look at Ubuntu 8.04 alpha 5. \hfill February 2008\\
\url{https://arstechnica.com/information-technology/2008/02/wubi-arrives-a-look-at-ubuntu-8-04-alpha-5/}\\

\textbf{Lifehacker} - Install Ubuntu on a Windows Netbook, No Partitioning Needed. \hfill May 2008\\
\url{https://lifehacker.com/install-ubuntu-on-a-windows-netbook-no-partitioning-ne-5542387}\\

\textbf{PCWorld} - The Ubuntu guide for displaced Windows users. \hfill March 2013\\
\url{https://www.pcworld.com/article/2030132/the-ubuntu-guide-for-displaced-windows-users.html}\\


%\null
%\vspace*{\fill}
%\hspace{-0.25\linewidth}\colorbox{white}{\makebox[1.5\linewidth][c]{\mystrut  \textnormal{\textcolor{sectcol}{www.jankuester.com} $\cdot$ \textcolor{sectcol}{github.com/jankapunkt}}}}




%============================================================================%
%
%
%
% DOCUMENT END
%
%
%
%============================================================================%
\end{document}
