%______________________________________________________________________________________________________________________
% @brief    LaTeX2e Resume for Geza Kovacs

\documentclass[margin,line]{resume}

\paperwidth 8.5in
\paperheight 11in

\usepackage{amssymb,amsmath}

\usepackage{paralist}

\usepackage{times}
\usepackage{hyperref}
\hypersetup{backref,
  pdftitle=Geza Kovacs Resume,
  pdfauthor=Geza Kovacs,
  pdfkeywords=Geza Kovacs Resume,
  pdfsubject=Geza Kovacs Resume,
  colorlinks=true,
  urlcolor=black} 


\usepackage{url}
%% Define a new 'leo' style for the package that will use a smaller font.
\makeatletter
\def\url@leostyle{%
\@ifundefined{selectfont}{\def\UrlFont{\sf}}{\def\UrlFont{
\small\rmfamily
}}}
\makeatother
%% Now actually use the newly defined style.
\urlstyle{leo}

\usepackage[absolute,overlay]{textpos}

\begin{document}

\vspace{-20mm}

\begin{textblock*}{17cm}(3mm,-2mm) % {block width} (coords) 
\begin{small}
\begin{center}
   Updated \today. Awards, publications, and teaching on next page. Latest at \href{http://www.gkovacs.com/resume.pdf}{http://www.gkovacs.com/resume}
\end{center}
\end{small}

\end{textblock*}

\vspace{-5.0mm}

\name{\LARGE{\textsf{Geza Kovacs}} \hspace{32.5mm} \large{\textsf{geza@cs.stanford.edu}} \hspace{35.5mm} \large{\textsf{\href{http://www.gkovacs.com}{gkovacs.com}}}}
\begin{resume}

\section{\mysidestyle Education}

\textbf{Stanford University} \vspace{0mm}\\\vspace{0mm}%
PhD, Computer Science  \hspace{2.5mm} GPA: 4.0/4.0 \hspace{30mm} Advisor: Michael Bernstein \hfill \textsl{2013 -- now}

\textbf{Massachusetts Institute of Technology} \vspace{0mm}\\\vspace{0mm}%
MEng, Computer Science \hspace{0mm} GPA: 4.9/5.0 \hspace{30mm} Advisor: Rob Miller \hfill \textsl{2012 -- 2013}\\\vspace{1mm}%
BS, Computer Science \hspace{4.5mm} GPA: 5.0/5.0 \hfill \textsl{2008 -- 2012}\vspace{-0.8mm}

\section{\mysidestyle Industry\\Experience}

%\textbf{Microsoft Research -- Research Intern, Redmond} \hspace{13.5mm} Mentor: Merrie Morris \hfill \textsl{Summer 2015}\\
\textbf{Microsoft Research -- Research Intern, Redmond} \hfill \textsl{Summer 2015}\\
Designed and build educational social feed for teaching literacy and mathematics. Published at CSCW 2017.
%Designed and built an educational social feed experience for teaching literacy and mathematics skills. Published as a full paper at CSCW 2017.

%\textbf{Microsoft Research -- Research Intern, Beijing} \hspace{17.5mm} Mentor: Darren Edge \hfill \textsl{Summer 2014}\\
\textbf{Microsoft Research -- Research Intern, Beijing} \hfill \textsl{Summer 2014}\\
Designed and built a quiz-directed lecture viewer to improve learners' engagement with in-video quizzes.
% Designed and implemented a quiz-directed lecture viewing system. User studies show it improves engagement and retention of questions compared to Coursera's in-video quizzes.

%\textbf{Google -- Software Engineering Intern, Mountain View} \hspace{5mm} Mentor: Shumin Zhai \hfill \textsl{Summer 2013}\\
\textbf{Google -- Software Engineering Intern, Mountain View} \hfill \textsl{Summer 2013}\\
Designed and built novel text input methods on Android phones and tablets.

\textbf{Google -- Software Engineering Intern, Mountain View} \hfill \textsl{Summer 2012}\\
Designed and built a system to detect and provide definitions for specialized vocabulary in books. %, by extracting them from the book text.

\textbf{Google -- Software Engineering Intern, Mountain View} \hfill \textsl{Summer 2011}\\
Developed a system to predict the quality of user reviews on the Android Marketplace (now Google Play). %, based on textual and metadata features. It was deployed and used for ordering reviews on Google Play.
% In user tests I conducted in several languages, the reviews selected by this algorithm
% were strongly preferred over chronological ordering.

\textbf{Microsoft Corporation -- Software Development Engineer Intern, Redmond} \hfill \textsl{Summer 2010}\\
\textbf{Google -- Summer of Code, FFmpeg (video transcoding library)} \hfill \textsl{Summer 2009}\\

% \textbf{Microsoft Corporation -- Software Development Engineer Intern, Redmond} \hfill \textsl{Summer 2010}\\
% Implemented the Intellisense API and Visual Studio code completion plugin for a new programming language. %under development by the Technical Computing group.

% \textbf{Google Summer of Code -- FFmpeg (Video transcoding library)} \hfill \textsl{Summer 2009}\\
% Developed a playlist and concatenation API and parsers for several playlist formats for FFmpeg.
%Developed a playlist and concatenation API parsers for several playlist formats, and a transitional interface for existing applications, for the FFmpeg video transcoding library.

\vspace{-3mm}

\section{\mysidestyle Research Experience}

\textbf{Stanford HCI Group -- PhD student}. Leading the following research projects: \hfill \textsl{Fall 2013 -- now}

\vspace{-2mm}

HabitLab: Personalized Interventions for Better Online Habits (published at \textbf{CHI 2019} and \textbf{CSCW 2018})\\
HabitLab is a Chrome extension and Android app which helps users achieve goals like reducing time on Facebook/Youtube, by deploying various interventions and determining which are most effective for users. \emph{12,000+ daily active users,} \url{http://habitlab.stanford.edu/}
 %This will help us gain insights about the effectiveness of classes

%\textit{TMI: Gaining Insights about How Users Behave Online}\\
%TMI is an volunteer-science project where users who install our Chrome extension can learn insights about how they use the web and compare to their peers, in exchange for contributing anonymized data to our dataset.

%\vspace{-2mm}

%FeedLearn: Microlearning in Facebook Feeds\\ %\hfill \textsl{Stanford HCI group, ongoing} \\
%\vspace{-3mm}
%\begin{compactitem}
% \textit{FeedLearn: Microlearning in Facebook Feeds}\\
%\textit{Edvertisements: Repurposing Web Advertisements as Microlearning Exercises}\\
%Edvertisements is a Chrome extension that helps you learn vocabulary as you browse the web, by replacing advertisements with microlearning exercises.
% FeedLearn helps you learn vocabulary as you browse your Facebook feed, by inserting interactive quizzes which you can answer without leaving your feed. User studies show increased vocabulary retention and engagement with quizzes, compared to the email and link approaches used by Duolingo.

\vspace{-2mm}

EduFeed: A Social Feed to Engage Preliterate Children in Educational Activities (published at \textbf{CSCW 2017})\\
Effects of In-Video Quizzes on MOOC Lecture Viewing (published at \textbf{L@S 2016})\\
FeedLearn: Microlearning in Facebook Feeds (published at CHI 2015 EA)\\
%FeedLearn is a Chrome extension that helps you learn vocabulary as you browse your Facebook feed, by inserting interactive quizzes which you can answer without leaving your feed. % User studies show increased vocabulary retention and engagement with quizzes, compared to the email and link approaches used by Duolingo.
QuizCram: Question-Driven Video Viewing (published at CHI 2015 EA) %\hfill \textsl{Microsoft Research, 2014} \\
%QuizCram is a viewer for MOOC lectures that enables quiz-driven video navigation and reviewing. User studies show higher engagement with quizzes and more reviewing compared to Coursera's in-video quiz format. % Materials can be generated from existing in-video quizzes on Coursera. % User studies show improved engagement with quizzes, increased reviewing, and improved test scores compared to Coursera's in-video quiz format.

%\end{itemize}
%
%\textbf{MIT CSAIL -- User Interface Design Group} \hfill \textsl{Fall 2011 -- Spring 2013}
%\vspace{-4.5mm}
%\begin{itemize}
%\begin{itemize}

\textbf{MIT UID Group -- Undergraduate/MEng research}. Led the following projects: \hfill \textsl{Fall 2011 -- Spring 2013}

\vspace{-2mm}

Smart Subtitles for Foreign Language Learning (published at \textbf{CHI 2014})\\ %\hfill \textsl{MIT CSAIL, UID group, 2013} \\ %\hfill \textsl{Spring 2012 -- Spring 2013} \\
%Smart Subtitles is a video viewer that uses an interactive transcript to help learners learn vocabulary while viewing foreign-language videos. Users learned more vocabulary with our system than with bilingual subtitles.
% Smart Subtitles helps you learn vocabulary while you watch foreign-language videos. It features an interactive transcript with mouse-over definitions and dialog-based navigation. User studies show increased vocabulary learning and increased satisfaction compared to bilingual subtitles.
%\vspace{-6mm}
%\vspace{-2mm}
%\vspace{-1mm}
%\vspace{-2mm}
%Full Paper at CHI 2014 (see publications section).\\
GrammarVis: Visualizing the Grammar of Foreign Languages (published at UIST 2013 demo)\\ %\hfill \textsl{MIT CSAIL, UID group, 2013} \\
%GrammarVis lets users interactively explore the syntactic structure of sentences.
%\vspace{-6mm}
%\vspace{-2mm}
%I built it as part of a foreign-language manga reader application.\\
%Developing a service that helps students learn vocabulary while watching foreign-language videos.
%It does so with ``smart subtitles'', which are transcripts annotated with information personalized to learners' proficiency levels, to help them understand new vocabulary in the dialog. %by providing user-customized annotations on subtitles for foreign
%\textit{To appear at the CHI 2013 Student Research Competition -- see Publications.}
ScreenMatch: Visual Context for Software Translators (published at CHI 2012 EA)\\ %\hfill \textsl{MIT CSAIL, UID group, 2012} \\  %\hfill \textsl{Fall 2011}\\
%Software translators lack visual context which illustrates how the strings they are translating are used.\\
%ScreenMatch matches translatable strings to screenshots, to illustrate how they are used in the software.

\vspace{-3mm}

%Developed a software internationalization tool which associates the translatable strings in an application with screenshots in which they appear (using OCR). By presenting a screenshot highlighting how the string is being used, this additional context allows translators to make more accurate translations.
%\textit{Presented at the CHI 2012 Student Research Competition, where it won first place -- see Publications.}
%\end{itemize}

%\emph{CHI 2012 student research competition, 1st place,}
%\end{itemize}
%\vspace{-4.5mm}

%\textbf{MIT Media Lab -- Affective Computing Group} \hfill \textsl{Spring 2009 -- Spring 2011}
%\begin{itemize}
%\item Developed a system to conduct mock interviews and highlight areas for improvement. I built the audio-related parts of the backend code and prototyped ways to visualize the speech transcript.
%\item Worked on a system to conduct mock interviews and highlight areas for improvement. I developed the audio-related parts of the backend code - for example, capturing the audio, detecting pauses, segmenting responses, and transcribing the speech. I also prototyped various ways to visualize the speech transcript.
%\item Created a library to allow scripts for the Praat acoustic analysis application to be programatically used with real-time, continuous streams of speech.
%\item Trained a classifier to determine mental states based on displacements of facial features, and used it in a demo application which performed real-time mental state classification.
%\end{itemize}

%\textbf{MIT CSAIL -- T-Party Project} \hfill \textsl{September -- December 2008}\\
%Implemented applications and user interfaces which seamlessly accumulate and present various forms of remotely-stored data on mobile Linux devices. The applications auto-configure and transparently utilize nearby networked peripherals such as external displays and audio devices.

\section{\mysidestyle Open-source\\Projects}

\textbf{UNetbootin (LiveUSB Creator)} \hfill \textsl{January 2007 -- now}\\
Built a utility to create bootable USB flash drives for a variety (50+) of Linux distributions.\\ %This work has been accepted into the official package repositories for Debian, Ubuntu, Fedora, openSUSE, Gentoo, and other major distributions. \\
\emph{40 million downloads,} \url{http://unetbootin.github.io/}

\textbf{Wubi (Ubuntu Installer for Windows)} \hfill \textsl{November 2006 -- August 2007}\\
Built the first versions of Wubi, which allows Windows users to safely install Ubuntu without repartitioning. \\ % This work is now part of Ubuntu. \\
% Built the first versions of the Windows-based Ubuntu Installer, which allows Windows users to safely install Ubuntu Linux without repartitioning. This work is now part of Ubuntu. \\
\emph{Now part of Ubuntu and ships on the official Ubuntu CD,} \url{http://wubi.sourceforge.net/}

\section{\mysidestyle Skills and technologies}

\textbf{Data Science}: Python, Numpy, SciPy, Pandas, Jupyter, R, rpy2, A/B testing, LMMs, ANOVA, Cox regression

\vspace{-4mm}

\textbf{Machine Learning and NLP}: PyTorch, TensorFlow, sklearn, RL, Deep Learning (CNN, RNN, GAN), PCFG % Weka

%\textbf{Machine Learning} and \textbf{Data Science}: PyTorch, TensorFlow, Python, Numpy, SciPy, Pandas, Jupyter, R, rpy2

\vspace{-4mm}

\textbf{Backend Development}: Node.js (Express, Koa), Flask, Django, MongoDB, PostgreSQL, Redis, EC2

\vspace{-4mm}

\textbf{Frontend Development}: HTML, CSS, JS, Web Components (Polymer), React, Angular, TypeScript, Flow
%\textbf{Web Development}: HTML, CSS, JavaScript, TypeScript, Node.js, Express, Flask, Polymer, React, MongoDB

\vspace{-4mm}

\textbf{Mobile Development}: Cross-platform (Cordova, NativeScript, React Native) and Android (Java)

\vspace{-4mm}

\textbf{Programming Languages}: Python, JavaScript, C, C++, C\#, Java, Ruby, CoffeeScript, LiveScript, Bash, R

\vspace{-4mm}

\textbf{Languages}: Fluent English and Chinese (Mandarin). Intermediate Hungarian, Vietnamese, Japanese, Spanish.

\pagebreak

%\section{\mysidestyle Programming}

%Have experience with Java, Python, C++, C\#, and web programming from projects and coursework.

% \section{\mysidestyle Publications}

%\textbf{Conference Papers}

\section{\mysidestyle Awards and Honors}

% Stanford Human-Centered AI Grant, 2018\\ %-2016\\
National Defense Science and Engineering Graduate Fellowship, 2013\\ %-2016\\
National Science Foundation Graduate Research Fellowship, 2013\\ %(declined in favor of NDSEG), 2013\\
Finalist and Honorable Mention, MIT Web Programming Competition (6.470), 2013\\ % (Project: PsetParty)
1$^{\textrm{st}}$ place, Most Useful, ACM UIST \begin{small}(User Interface Software and Technology)\end{small} Student Innovation Contest, 2012\\
1$^{\textrm{st}}$ place, ACM CHI \begin{small}(Conference on Human Factors in Computing Systems)\end{small} Student Research Competition, 2012\\
%1$^{\textrm{st}}$ place awards from 7digital and TokBox, Boston Music Hack Day, 2012\\ % (Project: InstantKaraoke)
1$^{\textrm{st}}$ place, MIT Autonomous Robotics Competition (Maslab), 2010
%Member of Tau Beta Pi (Engineering), Phi Beta Kappa (Liberal Arts), Eta Kappa Nu (EECS) honor societies

\section{\mysidestyle Teaching\\Experience}

\textbf{Teaching Assistant -- Understanding Users (CS 377U) at Stanford} \hfill \textsl{Spring 2019} \\

\vspace{-8mm}

\textbf{Teaching Assistant -- Human Computer Interaction Research (CS 376) at Stanford} \hfill \textsl{Fall 2018} \\
%Helped write and grade assignments, lead discussions, and manage the course infrastructure.

\vspace{-8mm}

\textbf{Teaching Assistant -- Natural Language Processing (6.863) at MIT} \hfill \textsl{Fall 2012} \\
%Helped write and grade assignments, and managed the course infrastructure. I developed new tools to make the assignment grading process faster, semi-automatic, and paper-free.

\vspace{-8mm}

\textbf{Instructor -- Introduction to C++ IAP (6.096) at MIT} \hfill \textsl{January 2011}\\

\vspace{-8mm}

%Gave lectures, helped write and grade assignments, and helped students in lab for a student-run, for-credit introductory C++ course. The teaching materials I produced have been made available on OpenCourseWare:
My lectures and teaching materials for this course are available on MIT OpenCourseWare:\\

\vspace{-8mm}

\url{http://ocw.mit.edu/courses/electrical-engineering-and-computer-science/6-096-introduction-to-c-january-iap-2011} \\

\vspace{-7mm}

\textbf{Software Director -- MASLAB Mobile Autonomous Systems Lab (6.186) at MIT} \hfill \textsl{January 2011}\\

\vspace{-8mm}

Gave lectures on computer vision and managed the software stack for this autonomous robotics competition.
% As the software director for this student-led autonomous robotics competition, I gave lectures on computer vision and control algorithms, managed the software stack for the competition, and helped students in lab.

\section{\mysidestyle Journal and Conference Papers}

% \section{\mysidestyle Publications: Full Papers}

\textbf{Geza Kovacs}, Drew Mylander Gregory, Zilin Ma, Zhengxuan Wu, Golrokh Emami, Jacob Ray, Michael Bernstein. ``Conservation of Procrastination: Do Productivity Interventions Save Time Or Just Redistribute It?'' ACM annual conference on Human Factors in Computing Systems (CHI) 2019. %Acceptance rate: 23.8\%

\textbf{Geza Kovacs}, Zhengxuan Wu, Michael Bernstein. ``Rotating Online Behavior Change Interventions Increases Effectiveness But Also Increases Attrition.'' ACM Conference on Computer-Supported Cooperative Work and Social Computing (CSCW) 2018. %Acceptance rate: 26\%

Rajan Vaish, Neil Gaikwad, \textbf{Geza Kovacs}, Andreas Veit, Ranjay Krishna, Imanol Arrieta Ibarra, Camelia Simoiu, Michael Wilber, Serge Belongie, Sharad Goel, James Davis, Michael Bernstein. ``Crowd Research: Open and Scalable University Laboratories.'' ACM Symposium on User Interface Software and Technology (UIST) 2017. %Acceptance rate: 22\%

Kiley Sobel, \textbf{Geza Kovacs}, Galen McQuillen, Andrew Cross, Nirupama Chandrasekaran, Nathalie Riche, Ed Cutrell, Meredith Morris. ``EduFeed: A Social Feed to Engage Preliterate Children in Educational Activities.'' ACM annual conference on Computer Supported Collaborative Work (CSCW) 2017. %Acceptance rate: 35\%

\textbf{Geza Kovacs}. ``Effects of In-Video Quizzes on MOOC Lecture Viewing.'' ACM annual conference on Learning at Scale (L@S) 2016. %Acceptance rate: 22\%

\textbf{Geza Kovacs} and Robert C. Miller. ``Smart Subtitles for Vocabulary Learning.'' ACM annual conference on Human Factors in Computing Systems (CHI) 2014. %Acceptance rate: 23\%

%\textbf{Extended Abstracts}

\section{\mysidestyle Peer-Reviewed Extended Abstracts}

Stanford Crowd Research, \textbf{Geza Kovacs}, Rajan Vaish, Michael Bernstein. ``Daemo: A Self-Governed Crowdsourcing Marketplace''. ACM Symposium on User Interface Software and Technology (UIST) 2015, Poster.

\textbf{Geza Kovacs}. ``FeedLearn: Using Facebook Feeds for Microlearning.'' ACM annual conference on Human Factors in Computing Systems (CHI) 2015, Extended Abstracts. % Acceptance rate: 45\%

\textbf{Geza Kovacs}. ``QuizCram: A Question-Driven Video Studying Interface.'' ACM annual conference on Human Factors in Computing Systems (CHI) 2015, Extended Abstracts.

Joseph Jay Williams, \textbf{Geza Kovacs}, Caren Walker, Samuel G Maldonado, Tania Lombrozo. ``Learning Online via Prompts to Explain.'' ACM annual conference on Human Factors in Computing Systems (CHI) 2014, Extended Abstracts.

\textbf{Geza Kovacs} and Robert C. Miller. ``Foreign Manga Reader: Learn Grammar and Pronunciation while Reading Comics.'' ACM Symposium on User Interface Software and Technology (UIST) 2013, Demo.

\textbf{Geza Kovacs}. ``Smart Subtitles for Language Learning.'' ACM annual conference on Human Factors in Computing Systems (CHI) 2013, Extended Abstracts.%\\
%\url{http://groups.csail.mit.edu/uid/other-pubs/chi2013-smartsubs.pdf}

\textbf{Geza Kovacs}. ``ScreenMatch: providing context to software translators by displaying screenshots.'' ACM annual conference on Human Factors in Computing Systems (CHI) 2012, Extended Abstracts.%\\
%\url{http://groups.csail.mit.edu/uid/other-pubs/chi2012-screenshots-for-translation-context.pdf}

%\vspace{-5mm}

\end{resume}
\end{document}
