%______________________________________________________________________________________________________________________
% @brief    LaTeX2e Resume for Geza Kovacs

\documentclass[margin,line]{resume}

\paperwidth 8.5in
\paperheight 11in

\usepackage{amssymb,amsmath}

\usepackage{times}
\usepackage{hyperref}
\hypersetup{backref,
  pdftitle=Geza Kovacs Resume,
  pdfauthor=Geza Kovacs,
  pdfkeywords=Geza Kovacs Resume,
  pdfsubject=Geza Kovacs Resume,
  colorlinks=true,
  urlcolor=blue} 


\usepackage{url}
%% Define a new 'leo' style for the package that will use a smaller font.
\makeatletter
\def\url@leostyle{%
\@ifundefined{selectfont}{\def\UrlFont{\sf}}{\def\UrlFont{
\small\rmfamily
}}}
\makeatother
%% Now actually use the newly defined style.
\urlstyle{leo}

\begin{document}

\vspace{-5.0mm}

\name{\LARGE{\textsf{Geza Kovacs}} \hspace{31.5mm} \large{\textsf{gkovacs@stanford.edu}} \hspace{31.5mm} \large{\textsf{(714) 251-6045}}}
\begin{resume}

\section{\mysidestyle Education}

\textbf{Stanford University} \vspace{0mm}\\\vspace{0mm}%
PhD, Computer Science\hfill \textsl{September 2013 -- present}

\textbf{Massachusetts Institute of Technology} \vspace{0mm}\\\vspace{0mm}%
MEng, Computer Science. GPA: 4.9/5.0 \hfill \textsl{September 2012 -- June 2013}\\\vspace{1mm}%
BS, Computer Science and Engineering. GPA: 5.0/5.0 \hfill \textsl{September 2008 -- June 2012}\vspace{-0.8mm}

\vspace{-0.5mm}

\section{\mysidestyle Research}

\textbf{MIT CSAIL -- User Interface Design Group}
%\vspace{-4.5mm}
%\begin{itemize}
\begin{itemize}
\item \textbf{Smart Subtitles for Foreign Language Learning}  \hfill \textsl{Spring 2012 -- Spring 2013} \\
Developing a service that helps students learn vocabulary while watching foreign-language videos.
It does so with ``smart subtitles'', which are transcripts annotated with information personalized to learners' proficiency levels, to help them understand new vocabulary in the dialog. %by providing user-customized annotations on subtitles for foreign
%\textit{To appear at the CHI 2013 Student Research Competition -- see Publications.}
\item \textbf{Providing Context to Software Translators by Displaying Screenshots}  \hfill \textsl{Fall 2011}\\
Developed a software internationalization tool which associates the translatable strings in an application with screenshots in which they appear (using OCR). By presenting a screenshot highlighting how the string is being used, this additional context allows translators to make more accurate translations.
%\textit{Presented at the CHI 2012 Student Research Competition, where it won first place -- see Publications.}
\end{itemize}

%\emph{CHI 2012 student research competition, 1st place,}
%\end{itemize}
%\vspace{-4.5mm}

\textbf{MIT Media Lab -- Affective Computing Group} \hfill \textsl{Spring 2009 -- Spring 2011}
\begin{itemize}
\item Worked on a system to conduct mock interviews and highlight areas for improvement. I developed the audio-related parts of the backend code - for example, capturing the audio, detecting pauses, segmenting responses, and transcribing the speech. I also prototyped various ways to visualize the speech transcript.
\item Created a library to allow scripts for the Praat acoustic analysis application to be programatically used with real-time, continuous streams of speech.
\item Trained a classifier to determine mental states based on displacements of facial features, and used it in a demo application which performed real-time mental state classification.
\end{itemize}

%\textbf{MIT CSAIL -- T-Party Project} \hfill \textsl{September -- December 2008}\\
%Implemented applications and user interfaces which seamlessly accumulate and present various forms of remotely-stored data on mobile Linux devices. The applications auto-configure and transparently utilize nearby networked peripherals such as external displays and audio devices.

\section{\mysidestyle Work\\Experience}

\textbf{Google Research -- Software Engineering Intern} \hfill \textsl{Summer 2013}\\
Working with the Input Methods Research team on mobile text entry.

\textbf{Google -- Software Engineering Intern} \hfill \textsl{Summer 2012}\\
Designed and implemented a system to detect and provide definitions for specialized vocabulary in books, by extracting them from the book text.

\textbf{Google -- Software Engineering Intern} \hfill \textsl{Summer 2011}\\
Developed a system that predicts how helpful a given user review on the Android Marketplace is.
In user tests I conducted in several languages, the reviews selected by this algorithm
were strongly preferred over chronological ordering. It has been deployed and is currently being used to display reviews on Google Play.

\textbf{Microsoft Corporation -- Software Development Engineer Intern} \hfill \textsl{Summer 2010}\\
Implemented the Intellisense API, refactoring options, and Visual Studio code completion plugin for a programming language under development by the Technical Computing group.

\textbf{Google Summer of Code -- FFmpeg (Video transcoding library)} \hfill \textsl{Summer 2009}\\
Developed a playlist and concatenation API, parsers for several playlist formats, and a transitional interface for existing applications, for the FFmpeg video transcoding library.

\section{\mysidestyle Open-source\\Projects}

\textbf{UNetbootin (LiveUSB creator)} \hfill \textsl{January 2007 -- present}\\
Created UNetbootin, a cross-platform utility to create bootable USB flash drives or perform network installations for a wide variety (50+) of Linux distributions. This work has been accepted into the official package repositories for Debian, Ubuntu, Fedora, openSUSE, Gentoo, and other major distributions. \\
\emph{20 million downloads,} \url{http://unetbootin.sourceforge.net/}

\textbf{Wubi (Windows-based Ubuntu Installer)} \hfill \textsl{November 2006 -- August 2007}\\
Designed and implemented the early versions of the Windows-based Ubuntu Installer, which allows Windows users to safely install Ubuntu Linux without repartitioning their hard drives. Formerly an independent project, this work is now part of Ubuntu. \\
\emph{Ships on the official Ubuntu CD,} \url{http://wubi.sourceforge.net/}

\pagebreak

\section{\mysidestyle Teaching}

\textbf{Teaching Assistant -- Natural Language Processing (6.863) at MIT} \hfill \textsl{Fall 2012} \\
Helped write assignments, managed the course infrastructure, and graded assignments. I developed new tools to make the assignment grading process faster, semi-automatic, and paper-free.

\textbf{Instructor -- Introduction to C++ IAP (6.096) at MIT} \hfill \textsl{January 2011}\\
Gave lectures, helped write and grade assignments, and helped students in lab for a student-run, for-credit introductory C++ course. The teaching materials I produced have been made available on OpenCourseWare:

\vspace{-4mm}

\url{http://ocw.mit.edu/courses/electrical-engineering-and-computer-science/6-096-introduction-to-c-january-iap-2011} \\

\vspace{-5mm}

\textbf{Software Director -- Maslab Autonomous Robotics Competition at MIT} \hfill \textsl{January 2011}\\
As the software director for the competition, I gave the software-related lectures,
managed the software for the competition, and helped students in lab.

%\section{\mysidestyle Programming}

%Have experience with Java, Python, C++, C\#, and web programming from projects and coursework.

\section{\mysidestyle Publications}

Geza Kovacs and Robert C. Miller. ``Foreign Manga Reader: Learn Grammar and Pronunciation while Reading Comics.'' ACM Symposium on User Interface Software and Technology (UIST) 2013, Demo (to appear).

Geza Kovacs. ``Smart Subtitles for Language Learning.'' ACM annual conference on Human Factors in Computing Systems (CHI) 2013, Extended Abstracts.%\\
%\url{http://groups.csail.mit.edu/uid/other-pubs/chi2013-smartsubs.pdf}

Geza Kovacs. ``ScreenMatch: providing context to software translators by displaying screenshots.'' ACM annual conference on Human Factors in Computing Systems (CHI) 2012, Extended Abstracts.%\\
%\url{http://groups.csail.mit.edu/uid/other-pubs/chi2012-screenshots-for-translation-context.pdf}

\section{\mysidestyle Awards}

National Defense Science and Engineering Graduate Fellowship, 2013-2016\\
NSF Graduate Research Fellowship (declined in favor of NDSEG), 2013 \\
%Finalist and Honorable Mention, MIT 6.470 Web Programming Competition 2013 (PsetParty)\\
%1$^{\textrm{st}}$ place awards from 7digital and TokBox, Boston Music Hack Day 2012 (InstantKaraoke)\\
1$^{\textrm{st}}$ place, Most Useful, ACM UIST \begin{small}(User Interface Software and Technology)\end{small} Student Innovation Contest 2012\\
1$^{\textrm{st}}$ place, ACM CHI \begin{small}(Conference on Human Factors in Computing Systems)\end{small} Student Research Competition 2012\\
1$^{\textrm{st}}$ place, MIT Maslab Autonomous Robotics Competition 2010

\vspace{-3mm}

\begin{small}
\begin{center}
Updated on \today. Latest version is at \url{http://gkovacs.github.com/resume.pdf}
\end{center}
\end{small}

\end{resume}
\end{document}
