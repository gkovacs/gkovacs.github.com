%______________________________________________________________________________________________________________________
% @brief    LaTeX2e Resume for Geza Kovacs

\documentclass[margin,line]{resume}

\paperwidth 8.5in
\paperheight 11in

\usepackage{times}
\usepackage{hyperref}
\hypersetup{backref,
  pdftitle=Geza Kovacs Resume,
  pdfauthor=Geza Kovacs,
  pdfkeywords=Geza Kovacs Resume,
  pdfsubject=Geza Kovacs Resume,
  colorlinks=true,
  urlcolor=blue} 


\usepackage{url}
%% Define a new 'leo' style for the package that will use a smaller font.
\makeatletter
\def\url@leostyle{%
\@ifundefined{selectfont}{\def\UrlFont{\sf}}{\def\UrlFont{
\small\rmfamily
}}}
\makeatother
%% Now actually use the newly defined style.
\urlstyle{leo}

\begin{document}

\vspace{-5.0mm}

\name{\LARGE{\textsf{Geza Kovacs}} \hspace{37.5mm} \large{\textsf{gkovacs@mit.edu}} \hspace{37.5mm} \large{\textsf{(714) 251-6045}}}
\begin{resume}

\section{\mysidestyle Education}

\textbf{Massachusetts Institute of Technology}, Cambridge, MA \vspace{0mm}\\\vspace{0mm}%
Master of Engineering in Computer Science\hfill \textsl{September 2012 -- June 2013}\\\vspace{1mm}%
Bachelor's Degree in Computer Science and Engineering \hfill \textsl{September 2008 -- June 2012}\vspace{-0.8mm}
\begin{itemize}
\item Overall GPA: 5.0/5.0 \\
\end{itemize}

\vspace{-6.5mm}

\section{\mysidestyle Work\\Experience}

\textbf{Google -- Software Engineering Intern} \hfill \textsl{June -- August 2012}\\
Developing a system to detect specialized vocabulary used in books,
and provide definitions for them by extracting them from the book text.

\textbf{Google -- Software Engineering Intern} \hfill \textsl{June -- August 2011}\\
Developed a model which would predict how helpful a given user review on the Android Marketplace is.
In user tests I conducted in several languages, the review ordering produced by this algorithm
was preferred over the existing one, and is thus currently being used on the site.

\textbf{Microsoft Corporation -- Software Development Engineer Intern} \hfill \textsl{June -- August 2010}\\
Designed and implemented the Intellisense API, refactoring options, and Visual Studio code completion plugin for a new programming language which is under development by the Technical Computing group.

\section{\mysidestyle Research}

\textbf{MIT CSAIL -- User Interface Design Group} \hfill \textsl{September 2011 -- present}\\
%\vspace{-4.5mm}
%\begin{itemize}
Developing a crowd-sourcing system to assist foreign language students in vocabulary acquisition while simultaneously generating annotated subtitles for videos. (MEng thesis)\\ %by providing user-customized annotations on subtitles for foreign-language videos.
Developed a software internationalization tool which associates the translatable strings in an application with screenshots in which they appear (using OCR). By presenting a screenshot highlighting how the string is being used, this additional context will allow translators to make more accurate translations.\\
%\emph{CHI 2012 student research competition, 1st place,}
\url{http://groups.csail.mit.edu/uid/other-pubs/chi2012-screenshots-for-translation-context.pdf}
%\end{itemize}
%\vspace{-4.5mm}

\textbf{MIT Media Lab -- Affective Computing Group} \hfill \textsl{February -- December 2009}\\
Trained a Bayesian network classifier to determine mental states from a still image or from a video stream based on displacements of facial features, and used it in a demo application which performed mental state classification in real-time from a webcam source.
Also created a library to allow scripts for the Praat acoustic analysis application to be programatically used with real-time, continuous streams of speech.

%\textbf{MIT CSAIL -- T-Party Project} \hfill \textsl{September -- December 2008}\\
%Implemented applications and user interfaces which seamlessly accumulate and present various forms of remotely-stored data on mobile Linux devices. The applications auto-configure and transparently utilize nearby networked peripherals such as external displays and audio devices.

\section{\mysidestyle Open-source\\Projects}

\textbf{UNetbootin (LiveUSB creator)} \hfill \textsl{January 2007 -- present}\\
Created UNetbootin, a cross-platform utility to create bootable USB flash drives or perform network installations for a wide variety (50+) of Linux distributions. This work has been accepted into the official package repositories for Debian, Ubuntu, Fedora, openSUSE, Gentoo, and other major distributions. \\
\emph{20 million downloads,} \url{http://unetbootin.sourceforge.net/}

\textbf{FFmpeg (Video transcoding library)} \hfill \textsl{May -- August 2009}\\
Designed and implemented a playlist and concatenation API, parsers for several playlist formats, and a transitional interface for existing applications, for the FFmpeg audio and video transcoder and library. This work was done as part of Google's Summer of Code program.

\textbf{Wubi (Windows-based Ubuntu Installer)} \hfill \textsl{November 2006 -- August 2007}\\
Designed and implemented the early versions of the Windows-based Ubuntu Installer, which allows Windows users to safely install Ubuntu Linux without repartitioning their hard drives. Formerly an independent project, this work is now part of Ubuntu. \\
\emph{Ships on the official Ubuntu CD,} \url{http://wubi.sourceforge.net/}

\section{\mysidestyle Teaching}

\textbf{Co-Instructor for Introduction to C++ IAP Course at MIT} \hfill \textsl{January 2011}\\
Taught an introductory C++ programming course for MIT students, giving lectures, reviewing assignments, and assisting students with software labs. Also helped archive the course on MIT's OpenCourseWare site:\\
\url{http://ocw.mit.edu/courses/electrical-engineering-and-computer-science/6-096-introduction-to-c-january-iap-2011}

%\section{\mysidestyle Programming}

%Have experience with Java, Python, C++, C\#, and web programming from projects and coursework.

\section{\mysidestyle Distinctions}

1$^{\textrm{st}}$ place, CHI 2012 \begin{small}(ACM Conference on Human Factors in Computing Systems)\end{small} Student Research Competition\\
1$^{\textrm{st}}$ place, Maslab 2010 (an autonomous robotics competition at MIT)

%\vspace{1mm}

\begin{small}
\begin{center}
Updated on \today. Latest version is at \url{http://gkovacs.github.com/resume.pdf}
\end{center}
\end{small}

\end{resume}
\end{document}
